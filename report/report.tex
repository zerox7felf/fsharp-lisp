\documentclass[11pt]{article}

\title{DVA229 Project - Simple Interpreter}
\author{Lucas Karlsson, Linus W. Monie}

\usepackage[utf8]{inputenc}
\usepackage[english]{babel}
\usepackage{caption}
\usepackage{subcaption}
\usepackage{graphicx}
\usepackage{listings}
\usepackage{float}
\usepackage{color, colortbl}
\usepackage{xcolor}
\usepackage{tabularx}

\graphicspath{ {./images/} }

%% Math symbols
\usepackage{amssymb}
\def\ojoin{\setbox0=\hbox{$\bowtie$}%
  \rule[-.02ex]{.25em}{.4pt}\llap{\rule[\ht0]{.25em}{.4pt}}}
\def\leftouterjoin{\mathbin{\ojoin\mkern-5.8mu\bowtie}}
\def\rightouterjoin{\mathbin{\bowtie\mkern-5.8mu\ojoin}}
\def\fullouterjoin{\mathbin{\ojoin\mkern-5.8mu\bowtie\mkern-5.8mu\ojoin}}

%% Code styling
\definecolor{codegreen}{rgb}{0,0.6,0}
\definecolor{codegray}{rgb}{0.5,0.5,0.5}
\definecolor{codepurple}{rgb}{0.58,0,0.82}
\definecolor{backcolour}{rgb}{0.95,0.95,0.92}

\lstdefinestyle{mystyle} {
	%% keywordstyle=\color{codepurple},
    %% ^- this gets weird, our syntax isn't super-lisp-ey, so highlights dont show up consistently.
	keywordstyle=\color{codegray},
	commentstyle=\color{codegray},
	%% stringstyle=\color{codegreen},
	basicstyle=\ttfamily\footnotesize,
	tabsize=2,
	showstringspaces=false,
}

\lstset{style=mystyle}

%% Font settings
\usepackage{helvet}
\newenvironment{codetext}{\ttfamily\selectfont}{\par}

%% Setup margins
\usepackage[centering, margin={1in, 0.5in}, includeheadfoot]{geometry}
\setlength{\topmargin}{0pt}
\setlength{\headsep}{0pt}
\setlength{\marginparsep}{11pt}

\begin{document}

\maketitle

\section*{Introduction}

    In this project we have implemented a simple interpreter for a completely functional, lisp-like programming language we call FsLisp. Like lisp it uses
    prefix notation and an expression-based syntax using parenthesis which closely mirror the generated abstract syntax tree. It is dynamically scoped, 
    uses lazy evaluation, is dynamically typed and supports higher-order functions.

\section{Syntax}
    
    All syntax rules follow the expression syntax:
    \lstinline{(fun arg arg ...)}\newline
    Expressions consist of two parenthesis with a number of tokens listed inside, separated by\newline whitespace. If there is more than one token,
    the first one will be evaluated with the others as arguments, otherwise the first token is evaluated and returned.
    
    \subsection{Available tokens}
        \begin{itemize}
            \item Booleans: either \lstinline{true} or \lstinline{false}.
            \item Integers: any positive whole number integer (note that negative integers are possible through the use of the unary minus built-in function, see built-in functions below).
            \item Strings: a string delimited by double quotes, whose contents can be any non-doble quote character or one of the escape strings
                  \lstinline{"\n"} (newline), \lstinline{"\""} (double quote), \lstinline{"\t"} (tab) or \lstinline{"\\"} (backslash).
            \item Identifiers: Any token which does not fall under any other syntax rules is an identifier, which when evaluated will return the function type bound to it (see types below). If this function type is first in an expression, it will be evaluated again, resulting in the underlying function being called. Otherwise, the function type itself is returned.
        \item Subexpressions: Any token inside an expression can itself be yet another expression.
        \end{itemize}

    \subsection{Types}

    FsLisp has a dynamic type system, meaning that expression arguments and return values can be of any type. There is no type-checking done while parsing,
    instead all typechecks are (optionally) done at runtime. The available types are: booleans, strings, signed integers, void (the result of an empty
    expression) and functions. The function type is used to represent function arguments to higher-order functions, as well as unnamed lambda expressions.

    \subsection{Built-in functions}
        \begin{tabularx}{\textwidth} { | >{\raggedright\arraybackslash}p{3cm} | >{\raggedright\arraybackslash}X | }
            \hline
            \lstinline{isFunc}, \lstinline{isBool},\newline \lstinline{isInt}, \lstinline{isStr}, \lstinline{isVoid} &
            Takes one argument and performs dynamic type checking on it, returning true if the argument is of the specified type. \\
            \hline
            \lstinline{+}, \lstinline{-}, \lstinline{*}, \lstinline{/} &
            Arithmetic operators, all taking two integers and returning the result. + and - both have special cases:
            \begin{itemize}
                \item \lstinline{-} can be unary, in which case the result is the same as “(- 0 arg)”
                \item \lstinline{+} can take strings instead of integers and will then return them concatenated.
            \end{itemize} \\
            \hline
            \lstinline{=}, \lstinline{>}, \lstinline{<} &
            Relational arithmetic operations. These take arguments of the types Boolean, Integer, String and Void. Their behaviour is left up
            to F\#’s arithmetic operations on the internal type-representations. \\
            \hline
            \lstinline{and}, \lstinline{or}, \lstinline{not} &
            Boolean arithmetic functions.\\
            \hline
            \lstinline{if} &
            Conditional function taking three arguments:
            \begin{enumerate}
                \item a condition of type boolean
                \item an expression to be evaluated if the condition is true
                \item an optional expression to be evaluated if the condition is false
            \end{enumerate}
            The function will return the result of the evaluated expression or, if the third argument is omitted and the condition is false,
            the value of the condition.\\
            \hline
            \lstinline{print} &
            Takes any number of arguments and prints them out in the console.\\
            \hline
            \lstinline{printDebug} &
            Prints out the internal type definition of all its arguments.\\
            \hline
            \lstinline{->} &
            Takes any number of arguments and evaluates them one by one in order, passing the scope returned by each expression to the next
            (see Implementation). \\
            \hline
            \lstinline{\$} &
            Constructs and returns lambda expressions. Takes any number of arguments (at least one), where the first arguments are strings
            which specify the names of the functions arguments, and the final argument is the expression to be evaluated when the function is called.\\
            \hline
            let &
            Binds an expression to an identifier. Takes two arguments, a string representing the name of the identifier and an expression to be evaluated
            when said identifier is evaluated. \\
            \hline
        \end{tabularx}

    \subsection{Other syntax rules}
        FsLisp supports C-style double-slash comments, which comment out all text from the start of the comment until the next newline character:
        \lstinline{//<comment>}

\newpage

\section{Implementation}
The FsLisp interpreter uses a recursive-descent parser to lexically analyze the source code and to then place the analyzed tokens into an abstract syntax tree. This syntax tree is then used by an evaluator which recursively translates identifiers into function values (given an initial scope in the form of a symbol table) and calls these functions with the sub-trees representing the arguments of the expression. It is then up to the individual functions to evaluate their arguments as needed.

\subsection{Abstract syntax tree}
    The structure of the styntax tree is dependent on the language's data structures for its types. These are organised into three nested groups:
    \begin{itemize}
        \item Syntax tokens: \newline
              These tokens describe tokens as lexed by the parser and directly represent values present in the source code:
              either identifiers, or:
        \item Value tokens: \newline
              These tokens describe actual values as used by and returned by the evaluator and the built-in functions. They can be either function
              values for representing higher-order functions, or:
        \item Constant values: \newline
              These can be of any of the other types available in the language, including the Void type, and contain their internal data.
              These are separated from function values as they represent data that can be operated on in ways that functions cannot, and vice versa.
    \end{itemize}
    The abstract syntax tree (AST) is defined by a recursive tree structure where each node can have one of three contents:
    \begin{itemize}
        \item A sub-tree containing a root node (itself represented by an AST type) along with a list of child-nodes (again recursively
              represented by AST types). The root node represents the first argument of an expression, while the list of child nodes represents
              its arguments.
        \item A value node containing either an identifier, a function value or a value constant.
        \item An empty tree.
    \end{itemize}

\subsection{Parser}

    The parser is implemented as a function which parses a single expression, and which recursively calls itself. It steps through the source code
    character by character and calls one of many helper functions depending on what it finds. Parsing of numbers, comments, strings and whitespace is all
    handled by these helper functions, which themselves call the parser function after completing parsing of their part of the syntax. The parser
    keeps a list of parsed tokens, which these helper functions may optionally add any parsed tokens to (for example, the string helper function
    adds a string constant token to this list before passing it to the parser). For any chars which are not handled by these helper functions,
    the parser adds it to a buffer and calls itself to keep parsing the next character.

    When the parser reaches a whitespace character, it takes
    any potential content in this buffer and creates an identifier syntax token out of it, placing it in the list of parsed tokens.
    When the parser reaches a closed parenthesis it takes the list of parsed tokens and builds an AST out of them. If there is more than
    one token it will return a sub-tree with the first node as the root.\newline Otherwise, if there is only one item in the expression, that item is
    returned as a value node. The final return values of the parser is this generated AST, along with input index, line and column numbers to
    indicate the position of where the parser stopped. If the parser reaches an open parenthesis in the expression it recursively calls itself to
    parse the contained subexpression, using the returned input index to skip over the sub-expression.

\subsection{Evaluator}

    The evaluator takes a symbol table as a scope and evaluates an AST node using it, returning its value in the form of a value token.
    If the node is a value node it does one of two things: if it is an identifier it looks up and returns the corresponding function,
    otherwise it simply returns the contained value token. If the node is a sub-tree it evaluates the root node and calls it with the child
    nodes as unevaluated arguments. If the root node evaluates to something other than a function, it returns an error message, as it cannot
    'call' a constant value. Finally, if the node is empty, the evaluator will return void.
    
    The symbol table used by the evaluator is a map of strings containing function names mapping to functions. These functions take a list of unevaluated
    AST nodes as arguments, along with a symbol table to act as its scope. These functions also return a value token as the value to be passed
    as a result in the language, and another symbol table to act as a new scope. It is left up to these functions to evaluate their arguments and
    to handle their internal scopes. Letting functions operate not only on data, but on the scope strucutre itself, is what lets parts of the
    language like function bindings through \lstinline{let} be implemented as functions. Passing these scopes from function to function also makes the
    implementation of a dynamic scope very simple. A function nested inside another function will receive the scope of the first function during
    evaluation, but the first function can disregard the scope returned by the nested function, letting it go out of scope. Functions like \lstinline{->}
    also become possible, which pass the scope of one evaluated argument to the evaluation of the next.

\subsection{Source files}

    The parser and evaluator are implemented in parser.fs and eval.fs respectively, while code for handling command-line arguments, the read-evaluate-print
    loop and interpreting files is in main.fs. All data structures beloning to the symbol table and abstract syntax tree can be found in types.fs.
    Function definitions are stored with the evaluator in eval.fs.

\section{User manual}

    For a description of the syntax, see Syntax above and example scripts provided in the sources.\newline
    The interpreter (FsLisp.exe) is run by specifying any number file names as command line arguments. These will then be run in sequence, unless
    no file names are given, in which case an interactive read-evaluate-print loop is started. The parser supports a few debugging options, which can
    be found by running the interpreter with the -h or --help command line option.

\newpage

\section{Final thoughts}

    The interpreter we made in this project is still at a prototype stage and could therefore be improved in many ways. Most important is probably speed:
    the Fibonacci sequence generator found in the included examples in the sourcecode takes considerable time, measurable in seconds(!). The code for
    generating new entries in the symbol table is also not very pretty and would benefit from better abstraction of handling of scopes and function
    arguments. The read-evaluate-print loop is also not very good, as it does not handle navigation keys at all. Worse, accidentally pressing them
    still stores them in memory even though they are not displayed properly, causing the parser to raise errors when passed the string. We are also
    not currently doing any checks on the function names passed to \lstinline{let} or \lstinline{$}. Because of this, the user can create functions which they later cannot
    access, which is not ideal.

\end{document}
